%EVERY VARIABLE HAS IT'S OWN PAGE
	<#if variable.distribution?? && variable.distribution.validResponses??>
		<#assign validResponseSize = variable.distribution.validResponses?size>
	<#else>
		<#assign validResponseSize = 0>
	</#if>

    \setcounter{footnote}{0}

    %omit vertical space
    \vspace*{-1.8cm}
	\section{${variable.name}<#if variable.label??> (${variable.label.de})</#if>}
	<#noescape>\label{section:${variable.name}}</#noescape>



	% TABLE FOR VARIABLE DETAILS
  % '#' has to be escaped
    \vspace*{0.5cm}
    \noindent\textbf{Eigenschaften\footnote{Detailliertere Informationen zur Variable finden sich unter
		\url{<#noescape>https://metadata.fdz.dzhw.eu/\#!/de/variables/${variable.id}</#noescape>}}}\\
	\begin{tabularx}{\hsize}{@{}lX}
	Datentyp: & <#if variable.dataType??>${variable.dataType.de!"-"}<#else>-</#if> \\
	Skalenniveau: & <#if variable.scaleLevel??>${variable.scaleLevel.de!"-"}<#else>-</#if> \\
	Zugangswege: &
	<#list variable.accessWays as accessWay>
	  ${displayAccessWay(accessWay)}<#sep>, </#sep>
	<#else>
	Keine Zugangswege!
	</#list> \\
	<#if variable.generationDetails?? && variable.generationDetails.description?? && variable.generationDetails.description.de??>
	  Generierungsbeschreibung: & ${removeMarkdown(variable.generationDetails.description.de)}\\
	</#if>
	<#if variable.annotations?? && variable.annotations.de??>
	  Anmerkungen: & ${removeMarkdown(variable.annotations.de)} \\
	</#if>
	<#if derivedVariables[variable.id]?has_content>
	  Abgeleitete Variablen: & <#list derivedVariables[variable.id] as derivedVariable>${derivedVariable.name}<#sep>, </#sep></#list> \\
	</#if>
	<#if repeatedMeasurementVariables[variable.id]?has_content>
	  Wiederholungsmessungen: & <#list repeatedMeasurementVariables[variable.id] as repeatedMeasurementVariable>${repeatedMeasurementVariable.name}<#sep>, </#sep></#list> \\
    </#if>

    \end{tabularx}



    %TABLE FOR QUESTION DETAILS
    %This has to be tested and has to be improved
    %rausfinden, ob einer Variable mehrere Fragen zugeordnet werden
    %dann evtl. nur die erste verwenden oder etwas anderes tun (Hinweis mehrere Fragen, auflisten mit Link)
	<#if variable.relatedQuestions?has_content>
		<#list variable.relatedQuestions as relatedQuestion>
			<#assign questionId = relatedQuestion.questionId>
			<#if questions[questionId]??>
				%TABLE FOR QUESTION DETAILS
				<#assign question = questions[questionId]>
				\vspace*{0.5cm}
                \noindent\textbf{Frage\footnote{Detailliertere Informationen zur Frage finden sich unter
		              \url{<#noescape>https://metadata.fdz.dzhw.eu/\#!/de/questions/${questionId}</#noescape>}}}\\
				\begin{tabularx}{\hsize}{@{}lX}
					Fragenummer: &
					<#if instruments[question.instrumentId]??>
					  ${instruments[question.instrumentId].description.de!"-"}:
					  ${question.number!"-"}
					<#else>
					  -
					</#if> \\
					%--
					Fragetext: & <#if relatedQuestion.relatedQuestionStrings??>${relatedQuestion.relatedQuestionStrings.de!"-"}<#else>-</#if> \\
				\end{tabularx}
			</#if>
		</#list>
	<#else>
		\vspace*{0.5cm}
		\noindent\textbf{Frage}\\
		Dieser Variable ist keine Frage zugeordnet.
	</#if>




	<#if variable.distribution?? && (variable.distribution.validResponses?? || variable.distribution.missings??)>
		<#if variable.scaleLevel??>

				%TABLE FOR THE NOMINAL / ORDINAL VALUES
        		\vspace*{0.5cm}
                \noindent\textbf{Häufigkeiten}

                \vspace*{-\baselineskip}
				<#if variable.dataType.de == "string">
					%STRING ELEMENTS NEEDS A HUGH FIRST COLOUMN AND A SMALL SECOND ONE
					<#noescape>\begin{filecontents}{\jobname-${variable.name!"-"}}</#noescape>
					\begin{longtable}{Xlrrr}
					\toprule
					\textbf{Wert} & \textbf{Label} & \textbf{Häufigkeit} & \textbf{Prozent (gültig)} & \textbf{Prozent} \\
					\endhead
					\midrule
					\multicolumn{5}{l}{\textbf{Gültige Werte}}\\
					<#if variable.distribution.validResponses??>
						%DIFFERENT OBSERVATIONS <=20
						<#if validResponseSize <= 20>
							<#list variable.distribution.validResponses as validResponse>

					\multicolumn{1}{X}{${validResponse.value!"-"}} &
					${(validResponse.label.de)!"-"} &
					\num{${validResponse.absoluteFrequency!"-"}} &
					\num[round-mode=places,round-precision=2]{${validResponse.validRelativeFrequency!"-"}} &
					\num[round-mode=places,round-precision=2]{${validResponse.relativeFrequency!"-"}} \\
					<#else>-\\

					</#list>
						%DIFFERENT OBSERVATIONS >20
						<#else>
							<#list firstTenValidResponses[variable.id] as validResponse>
								\multicolumn{1}{X}{${validResponse.value!"-"}} & ${(validResponse.label.de)!"-"} & \num{${validResponse.absoluteFrequency!"-"}} & \num[round-mode=places,round-precision=2]{${validResponse.validRelativeFrequency!"-"}} & \num[round-mode=places,round-precision=2]{${validResponse.relativeFrequency!"-"}} \\
							</#list>
							... & ... & ... & ... & ... \\
							<#list lastTenValidResponses[variable.id] as validResponse>
								\multicolumn{1}{X}{${validResponse.value!"-"}} & ${(validResponse.label.de)!"-"} & \num{${validResponse.absoluteFrequency!"-"}} & \num[round-mode=places,round-precision=2]{${validResponse.validRelativeFrequency!"-"}} & \num[round-mode=places,round-precision=2]{${validResponse.relativeFrequency!"-"}} \\
							</#list>
						</#if>
					<#else>
						& & 0 & 0 & 0 \\
					</#if>
					\midrule
					<#if variable.distribution.validResponses??>
						\multicolumn{2}{l}{Summe (gültig)} & \textbf{\num{${variable.distribution.totalValidAbsoluteFrequency!"-"}}} &
						\textbf{\num{100}} &
					    \textbf{\num[round-mode=places,round-precision=2]{${variable.distribution.totalValidRelativeFrequency!"-"}}} \\
					</#if>
					\multicolumn{5}{l}{\textbf{Fehlende Werte}}\\
					<#if variable.distribution.missings??>
						<#list variable.distribution.missings as missing>
							${missing.code!"-"} & ${(missing.label.de)!"-"} & \num{${missing.absoluteFrequency!"-"}} & - & \num[round-mode=places,round-precision=2]{${missing.relativeFrequency!"-"}} \\

						</#list>
					<#else>
						& & 0 & 0 & 0 \\
					</#if>
					\midrule
					\multicolumn{2}{l}{\textbf{Summe (gesamt)}} & \textbf{\num{${variable.distribution.totalAbsoluteFrequency!"-"}}} & \textbf{-} & \textbf{\num{100}} \\
					\bottomrule
					\end{longtable}
					\end{filecontents}
					<#noescape>\LTXtable{\textwidth}{\jobname-${variable.name!"-"}}</#noescape>

				<#else>
					%NUMERIC ELEMENTS NEED A HUGH SECOND COLOUMN AND A SMALL FIRST ONE
					<#noescape>\begin{filecontents}{\jobname-${variable.name!"-"}}</#noescape>
					\begin{longtable}{lXrrr}
					\toprule
					\textbf{Wert} & \textbf{Label} & \textbf{Häufigkeit} & \textbf{Prozent (gültig)} & \textbf{Prozent} \\
					\endhead
					\midrule
					\multicolumn{5}{l}{\textbf{Gültige Werte}}\\
					<#if variable.distribution.validResponses??>
						%DIFFERENT OBSERVATIONS <=20
						<#if validResponseSize <= 20>
							<#list variable.distribution.validResponses as validResponse>

					${validResponse.value!"-"} &
				% TODO try size/length gt 0; take over for other passages
					\multicolumn{1}{X}{<#if validResponse.label?? && validResponse.label.de?? && validResponse.label.de != ""> ${validResponse.label.de!"-"}  <#else> - </#if> } &


					%${validResponse.absoluteFrequency!"-"} &
					<#if validResponse.absoluteFrequency??>
					  \num{${validResponse.absoluteFrequency}} &
					<#else>
					  - &
					</#if>
					%--
					<#if validResponse.validRelativeFrequency??>
					  \num[round-mode=places,round-precision=2]{${validResponse.validRelativeFrequency}} &
					<#else>
					  - &
					</#if>
					<#if validResponse.relativeFrequency??>
					    \num[round-mode=places,round-precision=2]{${validResponse.relativeFrequency}} \\
					<#else>
					    -\\
					</#if>
							%????
							<#else>
								- \\
							</#list>
						%DIFFERENT OBSERVATIONS >20
						<#else>
							<#list firstTenValidResponses[variable.id] as validResponse>
								${validResponse.value!"-"} & \multicolumn{1}{X}{${(validResponse.label.de)!"-"}} & %${validResponse.absoluteFrequency!"-"} &
								<#if validResponse.absoluteFrequency??>
								  \num{${validResponse.absoluteFrequency}} &
								<#else>
								  - &
								</#if>
								%--
								<#if validResponse.validRelativeFrequency??>
								  \num[round-mode=places,round-precision=2]{${validResponse.validRelativeFrequency}} &
								<#else>
								  - &
								</#if>
								<#if validResponse.relativeFrequency??>
								  \num[round-mode=places,round-precision=2]{${validResponse.relativeFrequency}} \\
								<#else>
								  -\\
								</#if>
							</#list>
							... & ... & ... & ... & ... \\
							<#list lastTenValidResponses[variable.id] as validResponse>
								${validResponse.value!"-"} & \multicolumn{1}{X}{${(validResponse.label.de)!"-"}} & %${validResponse.absoluteFrequency!"-"} &
								<#if validResponse.absoluteFrequency??>
								  \num{${validResponse.absoluteFrequency}} &
								<#else>
								  - &
								</#if>
								%--
								<#if validResponse.validRelativeFrequency??>
								  \num[round-mode=places,round-precision=2]{${validResponse.validRelativeFrequency}} &
								<#else>
								  - &
								</#if>
								<#if validResponse.relativeFrequency??>
								  \num[round-mode=places,round-precision=2]{${validResponse.relativeFrequency}} \\
								<#else>
								  -\\
								</#if>

							</#list>
						</#if>
					<#else>
						& & \num{0} & \num{0} & \num{0} \\
					</#if>
					\midrule
					<#if variable.distribution.validResponses??>
					\multicolumn{2}{l}{Summe (gültig)} &
					<#if variable.distribution.totalValidAbsoluteFrequency??>
					  \textbf{\num{${variable.distribution.totalValidAbsoluteFrequency}}} &
					<#else>
					  - &
					</#if>
					\textbf{\num{100}} &
					<#if variable.distribution.totalValidRelativeFrequency??>
					  \textbf{\num[round-mode=places,round-precision=2]{${variable.distribution.totalValidRelativeFrequency}}} \\
					<#else>
					  -\\
					</#if>
					%--
					</#if>
					\multicolumn{5}{l}{\textbf{Fehlende Werte}}\\
					<#if variable.distribution.missings??>
						<#list variable.distribution.missings as missing>
							${missing.code!"-"} &
							${(missing.label.de)!"-"} &
							<#if missing.absoluteFrequency??>
							  \num{${missing.absoluteFrequency}} &
							<#else>
							  - &
							</#if>
							 - &
							<#if missing.relativeFrequency??>
							  \num[round-mode=places,round-precision=2]{${missing.relativeFrequency}} \\
							<#else>
							  -\\
							</#if>
						</#list>
					<#else>
						& & 0 & 0 & 0 \\
					</#if>
					\midrule
					\multicolumn{2}{l}{\textbf{Summe (gesamt)}} &
					<#if variable.distribution.totalAbsoluteFrequency??>
				      \textbf{\num{${variable.distribution.totalAbsoluteFrequency}}} &
				    <#else>
				      \textbf{-} &
				    </#if>
				    \textbf{-} &
				    \textbf{\num{100}} \\
					\bottomrule
					\end{longtable}
					\end{filecontents}
					<#noescape>\LTXtable{\textwidth}{\jobname-${variable.name!"-"}}</#noescape>
				<#noescape>\label{tableValues:${variable.name}}</#noescape>
				\vspace*{-\baselineskip}
                <#if variable.distribution.validResponses??>
                    <#assign measures = []/>
                    <#if variable.distribution.statistics.minimum??>
                      <#assign measures = measures + [{"name":"Minimum ($min$)",
                      	 "value":"${variable.distribution.statistics.minimum}"}] />
                    </#if>
                   	<#if variable.distribution.statistics.maximum??>
                   	  <#assign measures = measures + [{"name":"Maximum ($max$)",
                   	  	 "value":"${variable.distribution.statistics.maximum}"}] />
                   	</#if>
                   	<#if variable.distribution.statistics.meanValue??>
                      <#assign measures = measures + [{"name":"arithmetisches Mittel ($\\bar{x}$)",
                      	"value":"\\num[round-mode=places,round-precision=2]{${variable.distribution.statistics.meanValue}}"}] />
                   	</#if>
                   	<#if variable.distribution.statistics.median??>
                      <#assign measures = measures + [{"name":"Median ($\\tilde{x}$)",
                      	"value":"${variable.distribution.statistics.median}"}] />
                   	</#if>
                   	<#if variable.distribution.statistics.mode??>
                      <#assign measures = measures + [{"name":"Modus ($h$)",
                      	"value":"${variable.distribution.statistics.mode}"}] />
                   	</#if>
                   	<#if variable.distribution.statistics.standardDeviation??>
                      <#assign measures = measures + [{"name":"Standardabweichung ($s$)",
                      	"value":"\\num[round-mode=places,round-precision=2]{${variable.distribution.statistics.standardDeviation}}"}] />
                   	</#if>
                   	<#if variable.distribution.statistics.skewness??>
                      <#assign measures = measures + [{"name":"Schiefe ($v$)",
                      	"value":"\\num[round-mode=places,round-precision=2]{${variable.distribution.statistics.skewness}}"}] />
                   	</#if>
                   	<#if variable.distribution.statistics.kurtosis??>
                       <#assign measures = measures + [{"name":"Wölbung ($w$)",
                       	"value":"\\num[round-mode=places,round-precision=2]{${variable.distribution.statistics.kurtosis}}"}] />
                   	</#if>
                    \begin{noten}
                	    \note{} Deskriptive Maßzahlen:
                	    Anzahl unterschiedlicher Beobachtungen: ${validResponseSize}%
                	    <#if (measures?size>0)>; </#if>
                	    <#list measures as item>
                	      <#noescape>${item.name}: ${item.value}<#sep>; </#sep></#noescape>
                	    </#list>
                     \end{noten}

                </#if>
			</#if>
		</#if>
	</#if>
