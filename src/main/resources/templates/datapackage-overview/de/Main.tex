\documentclass[ngerman]{book}

\ifx\directlua\undefined\ifx\XeTeXcharclass\undefined
  \usepackage[utf8]{inputenc}                %pdftex engine
  \else\RequirePackage[no-math]{fontspec}\fi %xetex engine
  \else\RequirePackage[no-math]{fontspec}\fi %luatex engine

\usepackage[marginalspalte]{dsreport}

%formatting numbers
\usepackage[locale = DE]{siunitx}

%formatting dates
\usepackage[ngerman]{isodate}

%support bold siunitx formatted numbers in table cells
\usepackage{etoolbox}
\robustify\bfseries

%siunitx defaults
%numbers in german notation using thousand seperators
%zahlen werden auf zwei stellen gerundet, integer erhalten keine nachfolgenden nullen
%detect-weight and detect-inline-weight are used to make a single cell bold
\sisetup{%
	output-decimal-marker={,},
	group-separator={.},
	group-digits=integer,
	group-minimum-digits=4,
	detect-weight=true,
	detect-inline-weight=math,
        detect-all
}

\subject{Datenpaketübersicht}
\author{<#list dataPackage.dataCurators as person>${person.firstName}<#if person.middleName??> ${person.middleName}</#if> ${person.lastName}<#sep> / </#sep></#list>}
\title{${dataPackage.title.de}}
\version{Version ${version}}
\bibliogrAngaben{}

\begin{document}
\frontmatter
\maketitle

\mainmatter
\variablesmatter

\setlength{\hsize}{166mm}
\setlength{\headsep}{0.7cm}
%\setlength{\evensidemargin}{\the\oddsidemargin}
%\setlength{\oddsidemargin}{22mm}
%\setlength{\marginparwidth}{-5cm}
%\setlength{\marginparsep}{0mm}
%\setlength{\leftmargin}{0mm}

{\color{spot}\huge\noindent
Übersicht zum Datenpaket "`${dataPackage.title.de}"'
}
\markboth{\truncate{166mm}{${dataPackage.title.de}}}{Datenpaketübersicht}

\setlength{\extrarowheight}{0.2cm}
\begin{longtable}{|p{0.3\textwidth}|p{0.65\textwidth}|}
\hline
<#if dataPackage.studySeries??>\textbf{Studienreihe} & ${dataPackage.studySeries.de} \vspace{0.18cm} \\
\hline</#if>
\textbf{Institutionen} & <#list dataPackage.institutions as institution><#if institution.de??>${institution.de}<#else>${institution.en}</#if><#sep> \newline </#sep></#list> \vspace{0.18cm} \\
\hline
\textbf{Gefördert von} & <#list dataPackage.sponsors as sponsor><#if sponsor.de??>${sponsor.de}<#else>${sponsor.en}</#if><#sep> \newline </#sep></#list> \vspace{0.18cm} \\
\hline
\textbf{Projektmitarbeiter*innen} & <#list dataPackage.projectContributors as person>${person.firstName}<#if person.middleName??> ${person.middleName}</#if> ${person.lastName}<#sep>, </#sep></#list> \vspace{0.18cm} \\
\hline
\textbf{Datenkuratierung} & <#list dataPackage.dataCurators as person>${person.firstName}<#if person.middleName??> ${person.middleName}</#if> ${person.lastName}<#sep>, </#sep></#list> \vspace{0.18cm} \\
\hline
<#if dataPackage.tags.de?has_content>\textbf{Themen} & <#list dataPackage.tags.de as tag>${tag}<#sep>, </#sep></#list> \vspace{0.18cm} \\
\hline</#if>
\textbf{Erhebungsdesign} & ${dataPackage.surveyDesign.de} \vspace{0.18cm} \\
\hline
<#if surveyDataTypes?has_content>\textbf{Erhebungsdatentyp} & <#list surveyDataTypes as surveyDataType>${surveyDataType.de}<#sep>, </#sep></#list> \vspace{0.18cm} \\
\hline</#if>
<#if surveys?has_content>\textbf{Erhebungen} & <#list surveys as survey><#if surveys?size gt 1>\textbf{Erhebung ${survey?counter}:} </#if>${survey.title.de}<#sep> \newline </#sep></#list> \vspace{0.18cm} \\
\hline</#if>
<#if populationDescriptionMapDe?has_content>\textbf{Grundgesamtheit} & <#list populationDescriptionMapDe?keys as population><#if surveys?size gt 1>\textbf{Erhebung <#list populationDescriptionMapDe?api.get(population) as surveyNumber>${surveyNumber}<#sep>, </#sep></#list>:} \newline </#if>${population}<#sep> \newline </#sep></#list> \vspace{0.18cm} \\
\hline</#if>
<#if surveySampleMapDe?has_content>\textbf{Stichprobe} & <#list surveySampleMapDe?keys as sample><#if surveys?size gt 1>\textbf{Erhebung <#list surveySampleMapDe?api.get(sample) as surveyNumber>${surveyNumber}<#sep>, </#sep></#list>:} \newline </#if>${sample}<#sep> \newline </#sep></#list> \vspace{0.18cm} \\
\hline</#if>
<#if surveyMethodMapDe?has_content>\textbf{Erhebungsmethode} & <#list surveyMethodMapDe?keys as surveyMethod><#if surveys?size gt 1>\textbf{Erhebung <#list surveyMethodMapDe?api.get(surveyMethod) as surveyNumber>${surveyNumber}<#sep>, </#sep></#list>:} </#if>${surveyMethod}<#sep> \newline </#sep></#list> \vspace{0.18cm} \\
\hline</#if>
<#if surveyPeriodMap?has_content>\textbf{Feldzeit} & <#list surveyPeriodMap?keys as period><#if surveys?size gt 1>\textbf{Erhebung <#list surveyPeriodMap?api.get(period) as surveyNumber>${surveyNumber}<#sep>, </#sep></#list>:} </#if>\daterange{${period.start}}{${period.end}}<#sep> \newline </#sep></#list> \vspace{0.18cm} \\
\hline</#if>
<#if surveys?has_content>\textbf{Nettostichprobe} & <#list surveys as survey><#if surveys?size gt 1>\textbf{Erhebung ${survey?counter}:} </#if>n = \num{${survey.sampleSize}}<#sep> \newline </#sep></#list> \vspace{0.18cm} \\
\hline</#if>
<#if cleanedResponseRatesMap?has_content>\textbf{Rücklaufquote} & <#list cleanedResponseRatesMap?keys as surveyNumber><#if surveys?size gt 1>\textbf{Erhebung ${surveyNumber}:} </#if>\num[round-mode=places,round-precision=2]{${cleanedResponseRatesMap?api.get(surveyNumber)}} \%<#sep> \newline </#sep></#list> \vspace{0.18cm} \\
\hline</#if>
<#if surveyAnnotationsMapDe?has_content>\textbf{Anmerkungen zu den Erhebungen} & <#list surveyAnnotationsMapDe?keys as annotation><#if surveys?size gt 1>\textbf{Erhebung <#list surveyAnnotationsMapDe?api.get(annotation) as surveyNumber>${surveyNumber}<#sep>, </#sep></#list>:} \newline </#if>${annotation}<#sep> \newline </#sep></#list> \vspace{0.18cm} \\
\hline</#if>
<#if accessWaysMap?has_content>\textbf{Datenprodukte und Zugangswege} & <#list accessWaysMap?keys as dataProduct>${dataProduct}<#list accessWaysMap[dataProduct] as accessWay>${accessWay}<#sep>, </#sep></#list> <#sep> \newline </#sep></#list> \vspace{0.18cm} \\
\hline</#if>
<#if nonQualiDataSets?has_content>\textbf{Datensatzstruktur} & <#list nonQualiDataSets as dataSet>\textbf{<#if nonQualiDataSets?size gt 1>Datensatz ${dataSet.number}: </#if>${dataSet.type.de}<#if dataSet.format??> im ${dataSet.format.en}-Format</#if>} \newline <#list numberOfObservationsMap?api.get(dataSet.number)?keys as dataProduct>${dataProduct}<#list numberOfObservationsMap?api.get(dataSet.number)?api.get(dataProduct) as pair>${pair.key} (n = \num{${pair.value}})<#sep>, </#sep></#list><#sep> \newline </#sep></#list><#sep> \newline \newline </#sep></#list> \vspace{0.18cm} \\
\hline</#if>
<#if qualiDataSets?has_content>\textbf{Datenkollektion} & <#list qualiDataSets as dataSet>\textbf{<#if qualiDataSets?size gt 1>Kollektion ${dataSet?counter}: </#if><#if dataSet.description.de?has_content>${dataSet.description.de}<#else>${dataSet.description.en}</#if>} \newline <#list numberOfObservationsMap?api.get(dataSet.number)?keys as dataProduct>${dataProduct}<#list numberOfObservationsMap?api.get(dataSet.number)?api.get(dataProduct) as pair>${pair.key} (n = \num{${pair.value}})<#sep>, </#sep></#list><#sep> \newline </#sep></#list><#sep> \newline \newline </#sep></#list> \vspace{0.18cm} \\
\hline</#if>
<#if doi?has_content>\textbf{DOI} & \href{https://doi.org/${doi}}{${doi}} \vspace{0.18cm} \\
\hline</#if>
<#if dataPackage.annotations?? && dataPackage.annotations.de?has_content>\textbf{Anmerkungen} & ${removeMarkdown(dataPackage.annotations.de)} \vspace{0.18cm} \\
\hline</#if>
\textbf{Release Notes} & Die Release Notes zu dieser Version des Datenpakets sind \href{<#noescape>${baseUrl}/de/data-packages/${dataPackage.masterId}?version=${version}</#noescape>}{hier unter "Materialien zu diesem Datenpaket"} zu finden. \vspace{0.18cm} \\
\hline
\textbf{Publikationen zum Datenpaket} & Publikationen zu diesem Datenpaket können  \href{<#noescape>${baseUrl}/de/data-packages/${dataPackage.masterId}?version=${version}&type=related_publications</#noescape>}{hier unter "Verbundene Objekte"} nachgeschlagen werden. \vspace{0.18cm} \\
\hline
<#if dataPackage.additionalLinks?has_content>\textbf{Weiterführende Links} & <#list dataPackage.additionalLinks as link>\href{${link.url}}{<#if link.displayText??><#if link.displayText.de?has_content>${link.displayText.de}<#else>${link.displayText.en}</#if><#else>${link.url}</#if>}<#sep> \newline </#sep></#list>  \vspace{0.18cm} \\
\hline</#if>
\end{longtable}
\end{document}
